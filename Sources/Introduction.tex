%%%%%%%%%%%%%%%%%%%%%%%%%%%%%%%%%%%%%%%%%%%%%%%%%%%%%%%%%%%%%%%%%%%%%%%%
\chapter{Introduction}
%%%%%%%%%%%%%%%%%%%%%%%%%%%%%%%%%%%%%%%%%%%%%%%%%%%%%%%%%%%%%%%%%%%%%%%%

In the era of Industry 4.0, companies exploit modern technology to optimize traditional manufacturing and industrial processes. As human labor is expensive, companies search for new ways to minimize human assistance. Eyesight is one of the senses that could not be replaced easily for a long time. However, with recent advances in the field of computer vision, some human tasks can be done by robots that consist of camera systems that allow robots to perceive their surroundings. 

In 2015 scientists from Microsoft Research team proposed a deep learning model that surpassed human-level accuracy on a visual recognition task on the ImageNet 2012 classification dataset \cite{surp2015}. Since then, even better models were proposed \cite{resnet, efficientnet}. These high-performing image recognition models became the backbone of state-of-the-art object detection models, which are the key component of machine perception.

There are many areas where machine perception can be useful in industry. For instance, the models can be trained to detect faulty products soon enough to take corrective actions to ensure the products' quality. It is also a vital part of robotic arms that can carry out non-trivial tasks such as picking, sorting, and packing items. Another example of use can be observed in autonomous delivery navigation robots that can pick up trays of objects and move them around in a large warehouse~\cite{bmw}. 