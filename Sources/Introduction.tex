%%%%%%%%%%%%%%%%%%%%%%%%%%%%%%%%%%%%%%%%%%%%%%%%%%%%%%%%%%%%%%%%%%%%%%%%
\chapter{Introduction}
%%%%%%%%%%%%%%%%%%%%%%%%%%%%%%%%%%%%%%%%%%%%%%%%%%%%%%%%%%%%%%%%%%%%%%%%

In the era of Industry 4.0, companies exploit modern technology to optimize
traditional manufacturing and industrial processes. As human labor is expensive,
companies search for new ways to minimize human assistance. Eyesight is one of
the senses that could not be replaced easily for a long time. However, with
recent advances in the field of \bld{computer vision}, some tasks can be done by
robots. The robots usually consist of camera systems that allow robots to
perceive their surroundings and make decisions.

In 2015 scientists from Microsoft Research team proposed a \bld{deep learning}
model \cite{surp2015} that surpassed human-level accuracy on a visual
recognition task on the ImageNet 2012 classification dataset \cite{imagenet}.
Since then, even superior models were proposed \cite{resnet, efficientnet}.
These high-performing image recognition models became the basis of
state-of-the-art \bld{object detection} models, which are the key component of
machine perception as their task is to localize the objects in the scene.

There are many areas where machine perception is useful in industry. For
instance, the models can be trained to detect faulty products soon enough to
take corrective actions to ensure the products' quality. It is also a vital part
of robotic arms that can carry out non-trivial tasks such as picking, sorting,
and packing of items. Another example of use can be observed in autonomous
delivery navigation robots that can pick up trays of objects and move them
around in a larger area such as a warehouse or a car factory \cite{bmw}.

The main goal of the thesis is \bld{to research} and \bld{to summarize} some of
the state-of-the-art object detection methods based on deep learning approach.
The models are evaluated on various real-world datasets provided by
collaborating company SANEZOO EUROPE, which offers solutions to problems such
as AI robot guidance and 3D vision inspection of products. Different models are
compared using appropriate metrics for object detection tasks. The results are
then summarized and discussed. For future evaluations on arbitrary datasets,
\bld{we implement a benchmarking tool} that provides an interactive way to train
and evaluate models.

We start by introducing the object detection task and the evaluation metrics we
use to measure the object detection models' performance. Next, we define the key
components of deep learning models. After defining all relevant concepts, we
begin to describe the models we have chosen for our evaluation. Chapter
\ref{chap:evaluation} introduces the provided datasets, describes the expected
results, training settings, and finally summarizes the evaluation outcomes. The
next Chapter \ref{chap:tool} describes the use case, the design, and the
implementation of our benchmarking tool. Lastly, we recapitulate our
contributions and propose future work in the final Chapter
\ref{chap:conclusion}. We use \bld{bold text} to highlight important parts of
text such as newly introduced terms, important concepts or explanations.

