\chapter{Conclusion}
\begin{itemize}
    \item What have we done
          \begin{itemize}
              \item Conduct a research on object detection methods
              \item Choose one of each type with a baseline
              \item Description with introduction to relevant topis from DL
              \item Set hypotheses based on prior benchmarks
              \item Train and evaluated
              \item Results
              \item Implemented a benchmark tool for future evaluations
          \end{itemize}
\end{itemize}

One of the thesis's goals was to conduct research and evaluate state-of-the-art deep learning object detection models on datasets provided by the company SANEZOO EUROPE.
We have chosen the well-known model Faster R-CNN as a baseline. The next selected model is Cascade R-CNN, which is an extension to the Faster R-CNN. Both models represent a family of detectors called two-stage detectors. Finally, we have chosen a representative of one -stage detectors -- RetinaNet.

At the beginning of the work, we briefly introduced the main concepts of deep neural networks. Then we have described the model architectures with emphasis on understandability by gradually introducing new building blocks of the detectors.

Next, we have introduced the provided datasets. Unfortunately, since the datasets consist mainly of the company's customer products, only one dataset could be published. However, we provided illustrations and detailed characteristics. Based on the prior results, we have set the hypotheses about the performance of our models. Afterward, we have described the training process and its details. We have discussed the evaluation results dataset by dataset and eventually summed up the results at the end of Chapter 4.


\section{Future Work}