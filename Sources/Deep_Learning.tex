%%%%%%%%%%%%%%%%%%%%%%%%%%%%%%%%%%%%%%%%%%%%%%%%%%%%%%%%%%%%%%%%%%%%%%%%
\chapter{Deep Learning}
%%%%%%%%%%%%%%%%%%%%%%%%%%%%%%%%%%%%%%%%%%%%%%%%%%%%%%%%%%%%%%%%%%%%%%%%

% Neural networks since 60s
% Deep learning since 2006
% Definition
  % Multiple layers
  % Based on ANN
% Impact

In this chapter I briefly introduce the field of deep learning. I assume previous knowledge of basics of machine learning.
All the concepts in this chapter are very well described in a book written by the pioneers of deep learning Ian Goodfellow, 
Yoshua Bengio and Aaron Courville /cite{Goodfellow-et-al-2016}. The historical development of artficial neural networks and
deep learning is nicely overviewd in a journal article by another important figure of the field Juergen Schmidhuber 
\cite{DBLP:journals/corr/Schmidhuber14}.

The history of artificial neural network (ANN) can be dated back to 1940s /cite{McCulloch_1943}. But only since 2006, deep 
architectures of ANN have become very popular area of machine learning research /cite{DBLP:journals/corr/Schmidhuber14}. It was
when Hinton et al. introduced unsupervised pre-training of deep feedforward neural networks. By using this method on a deep belief
network [DBN CITATION], the model achiev 1.2% error rate on the MNIST classification dataset [MNIST CITATION]. This exceptional 
result helped to draw attention to deep belief networks.

\section{Definition}
So what is deep learning? There are many closely related abstract definitions, that are summarized in /cite{mic_definitions}. 
I cite the one that is most suitable for this work. 

Definition