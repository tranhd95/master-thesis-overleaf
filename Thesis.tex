\documentclass{mimosis}

% My packages
\usepackage{todonotes}
\usepackage{amsmath} % For definitions
\usepackage{threeparttable}
\usepackage{float}
\usepackage{metalogo}
\usepackage[T1]{fontenc}
%%%%%%%%%%%%%%%%%%%%%%%%%%%%%%%%%%%%%%%%%%%%%%%%%%%%%%%%%%%%%%%%%%%%%%%%
% Some of my favourite personal adjustments
%%%%%%%%%%%%%%%%%%%%%%%%%%%%%%%%%%%%%%%%%%%%%%%%%%%%%%%%%%%%%%%%%%%%%%%%
%
% These are the adjustments that I consider necessary for typesetting
% a nice thesis. However, they are *not* included in the template, as
% I do not want to force you to use them.

% This ensures that I am able to typeset bold font in table while still aligning the numbers
% correctly.
\usepackage{etoolbox}

\usepackage[binary-units=true]{siunitx}
\DeclareSIUnit\px{px}
% \newcommand{\todo}[1]{\textcolor{red}{(#1)}}
\newcommand{\bld}[1]{\textbf{#1}}

\sisetup{%
  detect-all           = true,
  detect-family        = true,
  detect-mode          = true,
  detect-shape         = true,
  detect-weight        = true,
  detect-inline-weight = math,
}


\usepackage[parfill]{parskip}

%%%%%%%%%%%%%%%%%%%%%%%%%%%%%%%%%%%%%%%%%%%%%%%%%%%%%%%%%%%%%%%%%%%%%%%%
% Hyperlinks & bookmarks
%%%%%%%%%%%%%%%%%%%%%%%%%%%%%%%%%%%%%%%%%%%%%%%%%%%%%%%%%%%%%%%%%%%%%%%%

\usepackage[%
  colorlinks = true,
  citecolor  = Green,
  linkcolor  = RoyalBlue,
  urlcolor   = RoyalBlue,
  unicode,
  ]{hyperref}

\usepackage{bookmark}

%%%%%%%%%%%%%%%%%%%%%%%%%%%%%%%%%%%%%%%%%%%%%%%%%%%%%%%%%%%%%%%%%%%%%%%%
% Bibliography
%%%%%%%%%%%%%%%%%%%%%%%%%%%%%%%%%%%%%%%%%%%%%%%%%%%%%%%%%%%%%%%%%%%%%%%%
%
% I like the bibliography to be extremely plain, showing only a numeric
% identifier and citing everything in simple brackets. The first names,
% if present, will be initialized. DOIs and URLs will be preserved.

\usepackage[%
  autocite     = plain,
  backend      = biber,
  doi          = true,
  url          = true,
  giveninits   = true,
  hyperref     = true,
  maxbibnames  = 99,
  maxcitenames = 99,
  sortcites    = true,
  style        = numeric,
  ]{biblatex}

\input{bibliography-mimosis}
\addbibresource{Thesis.bib}

%%%%%%%%%%%%%%%%%%%%%%%%%%%%%%%%%%%%%%%%%%%%%%%%%%%%%%%%%%%%%%%%%%%%%%%%
% Fonts
%%%%%%%%%%%%%%%%%%%%%%%%%%%%%%%%%%%%%%%%%%%%%%%%%%%%%%%%%%%%%%%%%%%%%%%%

\ifxetexorluatex
  \setmainfont{Minion Pro}
\else
  \usepackage[lf]{ebgaramond}
  \usepackage[oldstyle,scale=0.7]{sourcecodepro}
  \singlespacing
\fi

\renewcommand{\th}{\textsuperscript{\textup{th}}\xspace}

\newacronym {ANN}{ANN}{Artificial neural network}
\newacronym {CNN}{CNN}{Convolutional neural network}

% \newglossaryentry{LaTeX}{%
%   name        = {\LaTeX},
%   description = {A document preparation system},
%   sort        = {LaTeX},
% }

% \newglossaryentry{Real numbers}{%
%   name        = {$\real$},
%   description = {The set of real numbers},
%   sort        = {Real numbers},
% }

\makeindex
\makeglossaries




%%%%%%%%%%%%%%%%%%%%%%%%%%%%%%%%%%%%%%%%%%%%%%%%%%%%%%%%%%%%%%%%%%%%%%%%
% Incipit
%%%%%%%%%%%%%%%%%%%%%%%%%%%%%%%%%%%%%%%%%%%%%%%%%%%%%%%%%%%%%%%%%%%%%%%%

\title{Object Detection in Manufacturing Industry}
\subtitle{\textsc{Master's thesis}}
\author{Hai Duong Tran}

\begin{document}

\frontmatter
%   \begin{titlepage}

  \vspace*{1.5cm}
  \makeatletter

  \begin{center}
    \begin{LARGE}
        \textsc{Masaryk University}
    \end{LARGE}\\
    \begin{Large}
        \textsc{Faculty of Informatics}
    \end{Large}\\[1cm]
    \includegraphics[width=4cm, height=4cm] {fi_logo.pdf}\\[2cm]
    \begin{Huge}
      \@title
    \end{Huge}\\[1.25cm]
    %
    \begin{Large}
      \@subtitle
    \end{Large}\\[1.5cm]
    \begin{LARGE}
    %
    \@author
    \end{LARGE}
    \vfill
    {\hfill\large Brno 2020}
  \end{center}
  \makeatother
\end{titlepage}

\newpage
\null
\thispagestyle{empty}
\newpage

%   \chapter*{Declaration}

\noindent
Hereby I declare, that this paper is my original authorial work, which I
have worked out by my own. All sources, references and literature used or
excerpted during elaboration of this work are properly cited and listed in
complete reference to the due source.
\vfill
\textbf{Advisor:} doc. RNDr. Pavel Matula, Ph.D. \\
\textbf{Consultant:} Ing. Michal Pustka
%   \chapter*{Acknowledgements}
\noindent
Lorem ipsum dolor sit amet, consectetuer adipiscing elit. Temporibus autem quibusdam et aut officiis debitis aut rerum necessitatibus saepe eveniet ut et voluptates repudiandae sint et molestiae non recusandae. Proin in tellus sit amet nibh dignissim sagittis. Aliquam id dolor. Fusce dui leo, imperdiet in, aliquam sit amet, feugiat eu, orci. Sed vel lectus. Donec odio tempus molestie, porttitor ut, iaculis quis, sem. Nullam faucibus mi quis velit. Quisque porta. Integer pellentesque quam vel velit. Cum sociis natoque penatibus et magnis dis parturient montes, nascetur ridiculus mus.
%   \chapter*{Abstract}

\noindent
Lorem ipsum dolor sit amet, consectetuer adipiscing elit. Integer vulputate sem a nibh rutrum consequat. Maecenas ipsum velit, consectetuer eu lobortis ut, dictum at dui. Fusce consectetuer risus a nunc. Integer in sapien. Nemo enim ipsam voluptatem quia voluptas sit aspernatur aut odit aut fugit, sed quia consequuntur magni dolores eos qui ratione voluptatem sequi nesciunt. Integer tempor. Nulla non arcu lacinia neque faucibus fringilla. Etiam sapien elit, consequat eget, tristique non, venenatis quis, ante. Aliquam erat volutpat. Curabitur bibendum justo non orci. Cras pede libero, dapibus nec, pretium sit amet, tempor quis. Etiam quis quam.

\section*{Keywords}
\noindent
Lorem, ipsum, dolor, sit, amet

  \tableofcontents

\mainmatter

    \let\cleardoublepage\clearpage
    %%%%%%%%%%%%%%%%%%%%%%%%%%%%%%%%%%%%%%%%%%%%%%%%%%%%%%%%%%%%%%%%%%%%%%%%
\chapter{Introduction}
%%%%%%%%%%%%%%%%%%%%%%%%%%%%%%%%%%%%%%%%%%%%%%%%%%%%%%%%%%%%%%%%%%%%%%%%

In the era of Industry 4.0, companies use modern technology to optimize
traditional manufacturing and industrial processes. As human labor is expensive,
companies search for new ways to minimize human assistance. One of the senses
that could not be replaced easily for a long time is eyesight. However, with
recent advances in the field of \bld{computer vision}, some tasks can be done by
robots. The robots usually consist of camera systems that allow them to
perceive their surroundings and make decisions.

In 2015 scientists from Microsoft Research team proposed a \bld{deep learning}
model \cite{surp2015} that surpassed human-level accuracy on a visual
recognition task on the ImageNet 2012 classification dataset \cite{imagenet}.
Since then, even superior models were proposed \cite{resnet, efficientnet}.
These high-performing image recognition models became the basis of
state-of-the-art \bld{object detection} models which are the key component of
machine perception as their task is to localize the objects in the scene.

There are many areas in industry where machine perception is useful. For
instance, the models can be trained to detect faulty products soon enough to
take corrective actions to ensure the products' quality. It is also a vital part
of robotic arms that can carry out non-trivial tasks such as items picking,
sorting, and packing. Another example of use can be observed in autonomous
delivery navigation robots that can pick up trays of objects and move them
around in a larger area such as a warehouse or a car factory \cite{bmw}.

The main goal of the thesis is \bld{to research} and \bld{to summarize} some of
the state-of-the-art object detection methods based on deep learning approach.
The \bld{models are evaluated on various real-world datasets} provided by
collaborating company \bld{SANEZOO EUROPE}, which offers solutions to problems
such as AI robot guidance and 3D vision inspection of products. Different models
are compared using appropriate metrics for object detection tasks. The results
are then summarized and discussed. For future evaluations on arbitrary datasets,
\bld{we implement a benchmarking tool} that provides an interactive way to train
and evaluate models.

The thesis is divided into six chapters. We start by introducing the object
detection task and the evaluation metrics we use to measure the object detection
models' performance. Next, we define the key components of deep learning models.
After defining all relevant concepts, we describe the models we have
chosen for our evaluation. \myref{Chapter}{chap:evaluation} introduces the
provided datasets, describes the expected
results, training settings, and finally summarizes the evaluation outcomes.
\myref{Chapter}{chap:tool} describes the use case, the design, and the
implementation of our benchmarking tool. Lastly, we recapitulate our
contributions and propose future work in the final
\myref{Chapter}{chap:conclusion}. We use \bld{bold text} to highlight important
parts of text such as newly introduced terms, important concepts or explanations.


    %%%%%%%%%%%%%%%%%%%%%%%%%%%%%%%%%%%%%%%%%%%%%%%%%%%%%%%%%%%%%%%%%%%%%%%%
\chapter{Object Detection}
%%%%%%%%%%%%%%%%%%%%%%%%%%%%%%%%%%%%%%%%%%%%%%%%%%%%%%%%%%%%%%%%%%%%%%%%
Object detection aims to locate objects in a given image and assign them to their \bld{class} (also called category or label). For illustration see Figure \ref{fig:od}. Generally, the task can be broken down into three steps: \bld{informative region selection}, \bld{feature extraction}, and \bld{classification}. In the first step, the method has to determine regions to which we apply the feature extraction. The feature extractor outputs a semantic numerical representation for each selected region is then used to predict the target's class. Note that, these steps are not fixed and the pipeline can vary.

\begin{figure}[h]
    \centering
    \includegraphics[width=0.9\linewidth]{Sources/Figures/objectdetection.png}
    \caption{A high-level illustration of an object detection pipeline. Adapted from \cite{objectdetectionfigure}.}
    \label{fig:od}
\end{figure}

Traditionally, engineers had to hand-craft feature extractors using algorithms such as SIFT \cite{sift}, SURF \cite{surf}, or HOG \cite{hog}. These methods are combined with well-established classification algorithms such as Support Vector Machines (SVM) \cite{svm}. Since this approach needs manual designing, it can be very demanding.

Deep learning methods introduced an end-to-end learning approach, which means that the model only takes a given set of annotated images to learn to detect key features, localize the objects, and classify them. So the hard work is done mainly by the model itself. This approach outperforms the traditional pipelines by a significant margin with respect to illumination and viewpoint changes \cite{outperforming}. We describe deep learning in detail in the Chapter \ref{deep_learning_chapter}. But firstly, let us describe how to measure the performance of object detectors.

% \begin{figure}[h]
%     \centering
%     \includegraphics[width=\linewidth]{Sources/Figures/comparison.png}
%     \caption{Comparison of: (a) Traditional approach and (b) Deep learning approach. Adapted from \cite{comparison_illustration}. \todo{Zmenit obrazek, clovek mate.}}
%     \label{fig:comparison}
% \end{figure}

\section{Evaluation Metrics}
Firstly, let us define evaluation metrics for binary classification. Consider two classes we want to predict: \bld{positive (P)} and \bld{negative (N)}. Let
\begin{itemize}
    \item \bld{true positives (TP)} be correctly predicted P cases,
    \item \bld{true negatives (TN)} be correctly predicted N cases,
    \item \bld{false negatives (FN)} be all P cases we wrongly predicted as N,
    \item \bld{false positives (FP)} be all N cases we wrongly predicted as P.
\end{itemize}
Suppose we want to focus on the performance of P cases prediction. We define metrics \bld{precision} and \bld{recall} as follows:
$$
    \text{precision} = \frac{\text{TP}}{\text{TP} + \text{FP}}, \\
    \text{recall} = \frac{\text{TP}}{\text{TP} + \text{FN}}.
$$
Precision is also known as a positive predictive value, i.e., how many of the positive predictions (TP + FP) actually belong to P class. Recall is known as a true positive rate and it measures how many TP we have predicted out of all actual P cases (TP + FN). There is a trade-off relationship between these metrics. If we increase the precision (e.g., by setting more strict constraints for assigning TP to a prediction), the recall decreases as there would be more FN cases, and vice versa. 

For an object detection model evaluation, we can suppose that TP represents correctly detected object. Similarly, FP can be viewed as a wrongly detected object or duplicate detection. We would like to combine the precision and recall to a single composite score -- \bld{mean average precision (mAP)}. However, before we delve into its definition, let us first describe how to determine the TP and FP from the model's predictions.

Models output a list of predictions that are defined by \bld{bounding box} coordinates (the boundaries of the detected object), \bld{predicted class}, and \bld{confidence score} of the prediction. To determine if the detection is correct, we need to define \bld{Intersection-over-Union (IoU)}. IoU is defined as a proportion of the area of intersection (overlap between predicted bounding box and the ground-truth bounding box) and the area of union (see Figure \ref{fig:iou}):
$$
    \text{IoU} = \frac{\text{area of intersection}}{\text{area of union}}.
$$

\begin{figure}[h]
    \centering
    \includegraphics[width=0.6\linewidth]{Sources/Figures/iou.png}
    \caption{Illustration of two bounding boxes and the two types of areas they form.}
    \label{fig:iou}
\end{figure}
We simply declare the prediction as a TP if the IoU of the ground-truth bounding box and the predicted bounding box is greater than a fixed IoU threshold. Otherwise the prediction is considered as a FP. Now that we know how to determine the TP and FP cases, we can define mAP.

Mean average precision is an averaged \bld{average precision (AP)} over all classes. Let us define AP for some class $c$. We denote it as AP$_{c}$. For illustration, consider a simplified example: we collect all predictions for the class $c$ from all test images, and we rank them by the confidence score. Let \bld{positive predictions} be those predictions that have IoU > 0.5. If there are more positive predictions for a single object, only the prediction with a higher confidence score is considered to be a TP. See the example Table \ref{tab:ap}. In the definition of AP, precision is defined as the proportion of all predictions above the rank which are TP. Whereas recall is defined as the proportion of TP ranked above a given rank and all positive predictions.
\begin{table}[H]
\centering 
\begin{threeparttable}
\begin{tabular}{|c|c|c|c|c|c|}
\hline
\bld{rank} & \bld{IoU > 0.5} & \bld{confidence score} & \bld{TP/FP} & \bld{Precision} & \bld{Recall} \\
\hline
1 & yes & 98 \% & TP     & 1.00  & 0.2  \\
2 & yes & 88 \% & FP$^*$  & 0.50  & 0.2  \\
3 & yes & 78 \% & TP     & 0.67  & 0.4  \\
4 & yes & 75 \% & TP     & 0.75  & 0.6  \\
5 & no  & 60 \% & FP     & 0.60  & 0.6  \\
6 & yes & 59 \% & TP     & 0.67  & 0.8  \\
\hline
\end{tabular}
\begin{tablenotes}
      \small
      \item  $^*$ consider this example as a duplicate of the rank 1 example.
\end{tablenotes}
\caption{An example of predictions.}
\label{tab:ap}
\end{threeparttable}
\end{table}

Let us explain the calculation behind the third row. The precision is $2/3 \doteq 0.67$, because there are 2 TP out of 3 predictions above the third row (including the row). The recall is $2/5 = 0.4$, since there are 2 TP out of 5 positive predictions (IoU > 0.5). As we can see, the precision has a "zigzag" pattern as we go down the ranking. Whereas the recall is increasing. 

The AP$_c$ is defined as an \bld{area under the precision-recall curve} (see Figure \ref{fig:precisionrecall}). To reduce the impact of the zigzags, we interpolate the curve at each recall value $r$ by taking the maximum precision $p_\text{interp}$ measured "to the right":
$$
p_\text{interp}(r) = \max\limits_{\hat{r}:\hat{r} \geq r} p(\hat{r}),
$$
where $p(\hat{r})$ is the measured precision at the measured recall $\hat{r}$. We obtain a monotonically decreasing curve that can be numerically integrated as it is a piecewise constant (see the red dashed curve in Figure \ref{fig:precisionrecall}). We sample all unique recall values $R$ and compute AP$_c$ as a sum of rectangular blocks as follows:
$$
\text{AP}_c = \sum\limits^{\lvert R\rvert - 1}_{n = 0} (r_{n+1} - r_n) \cdot p_\text{interp}(r_{n+1}).
$$

\begin{figure}[H]
    \centering
    \includegraphics[width=0.8\linewidth]{Sources/Figures/precisionrecall.png}
    \caption{Precision-recall curves.}
    \label{fig:precisionrecall}
\end{figure}

We then get the final mAP by averaging all AP$_c$ for every class (let $C$ be the set of all classes):
$$
    \text{mAP} = \frac{1}{\lvert C \rvert} \cdot \sum_{ c\in C} \text{AP}_c.
$$
The VOC competitions \cite{voc} used this metric for a fixed IoU = 0.5 threshold. However, for our evaluation, we use the \bld{updated COCO challenge metric} \cite{coco} that averages the mAP over 10~different IoU thresholds (from 0.50 to 0.95 with 0.05 step size). We denote it as AP$_{.50:.05:.90}$ = AP. Similarly, for single IoU thresholds we have AP$_{50}$ (the standard VOC metric) and AP$_{75}$ (a~strict metric). They also define AP$_\text{small}$, AP$_\text{medium}$, and AP$_\text{large}$ for varying object's size:
\begin{itemize}
    \item AP$_\text{small}$ is an AP for small objects (bounding box area < $32^2$ px),
    \item AP$_\text{medium}$ is an AP for medium objects ($32^2$ < area < $96^2$ px),
    \item AP$_\text{large}$ is an AP for large objects (area > $96^2$ px).
\end{itemize}
Note that they do not make distinction between mAP and AP. Henceforth, whenever we refer to the AP, we mean the primary COCO metric AP$_{.50:.05:.90}$. 











    \chapter{Datasets}
We are provided with three datasets by the company SANEZOO. Unfortunately, two of the datasets cannot be published as they are protected under a confidentiality agreement. However, we provide sufficient characteristics of the datasets. We summarize the basic information of the datasets in Table \ref{tab:datasets}. Detailed description is provided below the table.\todo{Doplnit chybejici udaje.}

\begin{table}[h]  
\centering
\begin{tabular}{|l|l|l|l|l|}
\hline
\bld{Type}         & \bld{Difficulty}   & \bld{Size} & \bld{Classes} & \bld{Publishable} \\ \hline
 Candies           & Easy               & 150        & 1             & Yes \\
 Metal parts       & Medium             & 5790       & 7             & No  \\ 
 Medical supplies  & Hard               & \textit{tbd}        & \textit{tbd}           & No  \\ \hline
\end{tabular}
\caption{Datasets summary.}
\label{tab:datasets}
\end{table}

\bld{Candies dataset.} Only the first dataset is publishable; thus, we provide a visual example image in Figure \ref{fig:candies}. The images consist only of a single class -- a golden candy. The density of objects (how densely they cover the image) varies through the dataset (see Figure \ref{fig:candies}). Object occlusions can occur in dense images; therefore, the dataset is not trivial. However, compared to our other datasets, it is still a simpler dataset as it strictly contains only a single class without foreign objects. Therefore, we assign an easy difficulty to this dataset.


\begin{figure}[h]

\begin{subfigure}{0.5\textwidth}
\includegraphics[width=0.9\linewidth]{Sources/Figures/sparse.jpg} 
\caption{Sparse example}

\end{subfigure}
\begin{subfigure}{0.5\textwidth}
\includegraphics[width=0.9\linewidth]{Sources/Figures/dense.png}
\caption{Dense example}

\end{subfigure}

\caption{Examples of candies dataset.}
\label{fig:candies}
\end{figure}

\bld{Metal parts dataset.} The second dataset consists of images with small metal parts (such as shafts, rings, casings). See Figure \ref{fig:parts} for rough illustration. The objects are placed in a plastic bin. There is always only one class of objects on a single image. The images are taken from a top-down perspective, similarly as in Figure \ref{fig:candies}. However, they are taken from various angles. The density of objects also differ through the dataset as in the first dataset (you can imagine a pile of metal parts).

\begin{figure}[ht]
    \centering
    \includegraphics[height=0.35\linewidth]{Sources/Figures/metal_parts.jpg}
    \caption{An illustration of classes from metal parts dataset. Taken from \cite{parts}.} 
    \label{fig:parts}
\end{figure}

\bld{Medical supplies dataset} The last set is made up of sequences of images taken from an assembly line's top view. The images capture a manual assembly of packages that are consisted of medical items (such as bandages, plastic cups, or surgery tools). We assign a hard difficulty to this dataset because of the non-static environment. To be more specific, the packages are in motion in some images; therefore, the objects are slightly blurred. Moreover, occlusion of objects occurs in the image very often since the operator does the packing manually, and the objects are stacked on top of each other. Some objects are tough to distinguish if they are next to each other (imagine a pile of cotton balls). The images are taken from above the operator similarly as in the first dataset.





 
    %%%%%%%%%%%%%%%%%%%%%%%%%%%%%%%%%%%%%%%%%%%%%%%%%%%%%%%%%%%%%%%%%%%%%%%%
\chapter{Deep Learning}
%%%%%%%%%%%%%%%%%%%%%%%%%%%%%%%%%%%%%%%%%%%%%%%%%%%%%%%%%%%%%%%%%%%%%%%%

% Neural networks since 60s
% Deep learning since 2006
% Definition
  % Multiple layers
  % Based on ANN
% Impact

In this chapter I briefly introduce the field of deep learning. I assume previous knowledge of basics of machine learning.
All the concepts in this chapter are very well described in a book written by the pioneers of deep learning Ian Goodfellow, 
Yoshua Bengio and Aaron Courville /cite{Goodfellow-et-al-2016}. The historical development of artficial neural networks and
deep learning is nicely overviewd in a journal article by another important figure of the field Juergen Schmidhuber 
\cite{DBLP:journals/corr/Schmidhuber14}.

The history of artificial neural network (ANN) can be dated back to 1940s /cite{McCulloch_1943}. But only since 2006, deep 
architectures of ANN have become very popular area of machine learning research /cite{DBLP:journals/corr/Schmidhuber14}. It was
when Hinton et al. introduced unsupervised pre-training of deep feedforward neural networks. By using this method on a deep belief
network [DBN CITATION], the model achiev 1.2% error rate on the MNIST classification dataset [MNIST CITATION]. This exceptional 
result helped to draw attention to deep belief networks.

\section{Definition}
So what is deep learning? There are many closely related abstract definitions, that are summarized in /cite{mic_definitions}. 
I cite the one that is most suitable for this work. 

Definition
    \chapter{Models}
The first deep learning-based object detection models emerged after a breakthrough in ImageNet Large Scale Visual Recognition Challenge (ILSVRC) in 2012 when Alex Krizhevesky et al. proposed a deep and wide CNN classification model. The so-called AlexNet outperformed traditional state-of-the-art models.  This architecture was later exploited as a feature extractor for the CNN-based object detection model R-CNN, which also surpassed the previous best results on PASCAL VOC 2012. Since then, many CNN-based object detection models were introduced. Generally, they can be categorized into two types: one-stage detectors and two-stage detectors. The latter approach consists of a regional proposal step, followed by classification and bounding box regression. This means that the model initially generates interesting regions, which are then analyzed one-by-one. In contrast, one-stage detectors are based on a global regression and classification, requiring a single pass through the network, and thus they are usually faster but less accurate. A common property of both approaches is that we can easily substitute the models' feature extractor. 

For our comparison, we have chosen three state-of-the-art models: Faster R-CNN, Cascade R-CNN, and RetinaNet. The first one was proposed in 2015, and it is a well-known example of two-stage detectors. The model is considered to be well-established as it has a reasonable accuracy/speed tradeoff and is easy to modify. In our work, Faster R-CNN serves as a baseline for our comparison. The next one, Cascade R-CNN, is an extension to Faster R-CNN, which improves the performance by addressing some of its caveats. Finally, RetinaNet is a recent one-stage detector introduced by the Facebook AI Research team (FAIR), aiming to match the two-stage detectors' accuracy while preserving one-stage detectors' speed. We describe all these models in detail in the following sections.

\begin{itemize}
\item General architecture
\begin{itemize}
\item Backbone
\item FPN
\end{itemize}
\item Faster R-CNN
\item Cascade R-CNN
\item RetinaNet
\end{itemize}

    
    \chapter{Evaluation}
    \begin{itemize}
    \item Jak byly modely natrenovany?
    \item Jake hyperparametry byly zvoleny? (počet iterací, regularizace, ...)
    \item HW specs
    \item Grafy ztrátových funkcí
    \item Tabulka s metriky
    \item Slovní komentář k výsledkům
    \end{itemize}
    
    \chapter{Benchmark Tool}
    \begin{itemize}
    \item Vysvětlit usecase
    \item Design 
    \item Použité technologie
    \end{itemize}
    
    \chapter{Future Work}
    \begin{itemize}
    \item Návrhy na zlepšení, lepší metody, ...
    \item (Možná spojit s conclusion)
    \end{itemize}
    
    \chapter{Conclusion}


% This ensures that the subsequent sections are being included as root
% items in the bookmark structure of your PDF reader.
\bookmarksetup{startatroot}
\backmatter

  \begingroup
    \let\clearpage\relax
    \glsaddall
    % \printglossary[type=\acronymtype]
    % \newpage
    % \printglossary
  \endgroup

  \printindex
  \begingroup
    \sloppy
    % \RaggedRight %% needs package ragged2e
      \printbibliography
    \endgroup

\end{document}
